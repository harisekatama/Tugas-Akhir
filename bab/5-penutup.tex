\chapter{PENUTUP}
\label{chap:penutup}

Pada bab ini akan dipaparkan kesimpulan dari hasil pengujian yang akan menjawab dari permasalahan yang diangkat dari pelaksanaan tugas akhir ini. Pada bab ini juga diapaprkan saran mengenai hal yang dapat dilakukan untuk mengembangkan penelitian kedepannya.

% Ubah bagian-bagian berikut dengan isi dari penutup

\section{Kesimpulan}
\label{sec:kesimpulan}

Berdasarkan hasil pengujian yang dilakukan selama pelaksanaan tugas akhir ini adalah sebagai berikut:

\begin{enumerate}[nolistsep]

  \item Waktu \emph{delay} rata-rata terendah terdapat pada pengujian dengan mengirimkan JSON dengan 1 nilai dan ditransmisikan melalui WiFi. Waktu \emph{delay} rata-ratanya adalah sebesar 1,032374783 detik.

  \item Waktu \emph{delay} rata-rata tertinggi terdapat pada pengujian dengan mengirimkan JSON dengan 2 nilai dan ditransmisikan melalui Bluetooth. Waktu \emph{delay} rata-ratanya adalah sebesar 1,038895345 detik. 
  
  \item Kursi Roda cenderung menyimpang kearah kiri apabila digerakkan dengan kontrol maju dengan sudut penyimpangan rata-rata sebesar 2,788\textdegree.
  
  \item Kursi Roda cenderung menyimpang kearah kanan apabila digerakkan dengan kontrol mundur dengan sudut penyimpangan rata-rata sebesar 5,494\textdegree.

\end{enumerate}

\section{Saran}
\label{chap:saran}

Berdasarkan hasil yang diperoleh dari penelitian ini maka saran yang dapat dipertimbangkan untuk pengembangan lebih lanjut adalah sebagai berikut:

\begin{enumerate}[nolistsep]

  \item Kecepatan putar motor diatur melalui ESP32 dengan menggunakan \emph{button} ataupun potensiometer.

  \item Mencoba teknik transmisi data lain yang lebih efektif dan efisien.

\end{enumerate}
