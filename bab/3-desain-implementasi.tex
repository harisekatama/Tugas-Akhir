\chapter{METODOLOGI}
\label{chap:metodologi}

% Ubah bagian-bagian berikut dengan isi dari desain dan implementasi

Penelitian ini dilaksanakan sesuai dengan desain sistem berikut ini beserta implementasinya. Desain sistem merupakan konsep dari pembuatan dan perencangan infrastruktur dan kemudian diwujudkan dalam bentuk blok-blok alur yang harus dikerjakan.

\section{Deskripsi Sistem}
\label{sec:deskripsisistem}

Tugas akhir ini merupakan penelitian yang mengintegrasikan teknologi visi komputer agar dapat mengontrol gerak kursi roda. Secara umum penelitian kali ini akan menggunakan desain sistem sesuai dengan Gambar \ref{fig:Metodologi Penelitian}.

% Gambar 3.1
\begin{figure} [ht] \centering
    % Nama dari file gambar yang diinputkan
    \includegraphics[scale=0.8]{gambar/DeskripsiSistem.png}
    % Keterangan gambar yang diinputkan
    \caption{Blok Diagram Penelitian}
    % Label referensi dari gambar yang diinputkan
    \label{fig:Metodologi Penelitian}
\end{figure}

\subsection{Paket Data}
Untuk dapat menggerakkan kursi roda maka perlu mengirimkan perintah ke kontroler kursi roda. Pada tahap klasifikasi pose telah didapatkan perintah dasar untuk menggerakkan kursi roda, seperti maju, mundur, kanan, kiri, maupun stop. Perintah ini kemudian akan digabungkan dengan kecepatan maksimal menjadi satu \emph{command} atau paket data seperti yang dilihat pada Persamaan \ref{eq:paket-data}.

% Persamaan 3.1
\begin{equation}
  \label{eq:paket-data}
    Arah,Kecepatan
\end{equation}

\begin{description}
  \item[Arah] : Variabel dengan tipe data \emph{char} yang digunakan untuk mengatur arah gerak motor kursi roda
  \item[Kecepatan] : Variabel dengan tipe data \emph{char} yang digunakan untuk mengatur kecepatan putar motor kursi roda 
\end{description}

Variabel arah memiliki tipe data \emph{char} yang akan menentukan gerak dari motor kursi roda, serta variabel kecepatan memiliki tipe data \emph{char} yang akan menentukan kecepatan maksimal dari kursi roda. Untuk memperkecil ukuran data maka kode instruksi untuk menentukan arah gerak dan kecepatan maksimal akan diwakili oleh satu huruf. Kode instruksi dapat dilihat pada Tabel \ref{tbl:kode-instruksi} dan Tabel \ref{tbl:kodePWM}. 

% Tabel 3.2
\begin{longtable}{|c|c|}
    \caption{Kode instruksi dari hasil klasifikasi}
    \label{tbl:kode-instruksi}\\
        \hline
        Klasifikasi Pose & Kode Instruksi \\ \hline
        \endfirsthead
        %
        \endhead
        %
        Kiri             & A              \\ \hline
        Maju             & B              \\ \hline
        Stop             & C              \\ \hline
        Mundur           & D              \\ \hline
        Kanan            & E              \\ \hline
\end{longtable}


\begin{longtable}{|c|c|}
  \caption{Kode instruksi untuk mengatur tingkat PWM}
  \label{tbl:kodePWM}\\
  \hline
  Kecepatan Maksimal & Kode Instruksi \\ \hline
  \endfirsthead
  %
  \endhead
  %
  0                     & O              \\ \hline
  31                    & P              \\ \hline
  63                    & Q              \\ \hline
  95                    & R              \\ \hline
  127                   & S              \\ \hline
  159                   & T              \\ \hline
  191                   & U              \\ \hline
  223                   & V              \\ \hline
  255                   & W              \\ \hline
  \end{longtable}

Setelah kedua variabel tersebut digabungkan maka akan dikirim secara nirkabel, baik menggunakan Bluetooth maupun WiFi dari laptop atau Jetson Nano ke ESP32.

\subsection{Memecahkan Paket Data}
Paket data yang telah dikirimkan melalui laptop maupun Jetson Nano akan diterima oleh ESP32 menggunakan Bluetooth maupun WiFi. Saat diterima oleh ESP32, data tersebut akan menjalani serangkaian proses yang melibatkan pemecahan paket data dan penyesuaian sesuai dengan variabel yang telah ditentukan sebelumnya. Pemecahan paket data ini memungkinkan ESP32 untuk mendekomposisi informasi yang terkandung dalam setiap paket dan memastikan bahwa setiap variabel terpisah dengan akurat. Dengan demikian proses ini akan mengorganisir dan menyusun kembali informasi serta memastikan bahwa setiap variabel telah benar sesuai dengan nama variabel dan tipe data yang disediakan.

\subsection{Kontrol Navigasi}
Kedua variabel yang didapatkan dari pemecahan paket data akan diproses pada ESP32. Variabel arah akan berperan untuk menentukan arah gerak dari motor, sedangkan variabel kecepatan akan digunakan untuk menetapkan kecepatan maksimal dari pergerakan motor tersebut. Terdapat serangkaian logika \emph{if} berantai pada kontrol navigasi, dimana empat variabel dir akan menentukan arah putaran motor. Selain itu, nilai PWM maksimal dikonfigurasi dengan menggunakan variabel kecepatan sehingga pengguna dapat menyesuaikan kecepatan maksimal motor yang pengguna inginkan. Dengan demikian pada tahap ini ESP32 dapat secara efektif memproses data yang diterima melalui sistem nirkabel dan menghasilkan instruksi kontrol yang sesuai untuk menggerakkan kursi roda dengan arah dan kecepatan yang diinginkan.


\section{Implementasi Alat}
\label{sec:implementasi alat}

Pada penelitian ini dikembangkan suatu alat kontrol yang dapat menerima perintah melalui perangkat lain seperti laptop maupun Jetson Nano secara nirkabel. Pada sub bab ini akan dijabarkan implementasi dari alat yang dikembangkan pada penelitian ini.

\subsection{\emph{Hardware} dan \emph{Software} yang digunakan}
Berikut ini dijabarkan beberapa \emph{Hardware} dan juga \emph{Software} yang digunakan pada penelitian ini seperti berikut:
\begin{enumerate}[nosep]
    \item Anaconda Navigator
    \item Arduino IDE 
    \item Laptop 
    \item Jetson Nano
    \item Kamera
    \item ESP32 Devkit V1
    \item 2 Kontroller Motor
    \item 2 DC-DC Voltage Regulator
    \item 2 DC Motor
    \item Baterai 24V
\end{enumerate}

\subsection{Skematik Alat}

Skematik pada alat ini diilustrasikan secara rinci pada Gambar \ref{fig:Skematik Alat}. Sistem ini menggunakan kamera yang dihubungkan dengan Laptop atau Jetson Nano sebagai perangkat utama dalam menangkap citra. Proses kerja dimulai saat kamera menangkap citra objek. Citra yang telah ditangkap ini lantas diproses oleh Laptop atau Jetson Nano. Di dalam sistem ini, model klasifikasi yang telah diprogram sebelumnya memainkan peran penting dalam menginterpretasikan data citra tersebut. Hasil dari proses klasifikasi ini sangat krusial karena menjadi dasar dalam penentuan kode instruksi yang akan diimplementasikan.

Kode instruksi tersebut kemudian akan dikombinasikan dengan parameter kecepatam maksimal yang sebelumnya telah ditetapkan oleh pemngguna. Gabungan dari kode instruksi dan parameter kecepatan ini akan menjadi satu paket data sebagai kontrol gerak kursi roda. Paket data ini kemudian akan ditransmisikan secara nirkabel, baik dengan Bluetooth maupun WiFi, ke modul ESP32 Devkit V1. 

\begin{figure} [ht] \centering
  % Nama dari file gambar yang diinputkan
  \includegraphics[scale=0.3]{gambar/bab3/Schematics.png}
  % Keterangan gambar yang diinputkan
  \caption{Skematik kontrol motor kursi roda}
  % Label referensi dari gambar yang diinputkan
  \label{fig:Skematik Alat}
\end{figure}

ESP32 memiliki peranan penting dalam kontrol motor kursi roda. ESP32 akan digunakan sebagai pusat pengendalian yang menerima paket data yang telah dikirimkan oleh pengguna secara nirkabel. Kemudian ESP32 akan melakukan pemecahan paket data dan menyesuaikan data tersebut kedalam variabel-variabel yang telah ditentukan. Proses pemecahan paket data ini akan menghasilkan dua data utama yang kemudian akan diproses lebih lanjut oleh ESP32.

Variabel pertama merupakan variabel arah yang memiliki fungsi krusial untuk menentukan arah gerak kedua motor pada kursi roda. Variabel ini akan memastikan bahwa motor bergerak sesuai dengan arah yang diinginkan sesuai dengan data yang diterima. Selain itu terdapat variabel kecepatan yang digunakan untuk menetapkan kecepatan maksimal pergerakan motor.


Dalam implementasinya terdapat serangkaian logika \emph{if} berantai yang akan dijelaskan secara terperinci pada sub bab program. Sederhananya logika \emph{if} berantai ini memiliki peran yang penting dalam pengambilan keputusan baik arah dan kecepatan motor sesuai dengan data yang telah diterima. Hasil dari logika ini akan memberikan \emph{trigger} berupa tegangan 5V ataupun 0V. Tegangan ini kemudian akan mempengaruhi arah putar motor.

Selanjutnya variabel kecepatan akan digunakan untuk mengatur tingkat \emph{Pulse Width Modulation} pada kontroler motor. Pengaturan PWM ini sangatlah penting guna mengatur kecepatan putar motor. Dengan mengatur tingkat PWM maka kecepatan motor maksimal dapat disesuaikan sesuai dengan kebutuhan.


\section{Kode Program}
Pada sub bab ini akan dijabarkan kode program yang digunakan pada penelitian ini.

\subsection{Program Untuk Memeriksa MAC Address Pada ESP32}
Agar dapat terhubung dengan Bluetooth pada ESP32 maka kita harus mengetahui MAC Address dari ESP32. Berikut merupakan program untuk memeriksa MAC Address Pada ESP32 dapat dilihat pada Program \ref{lst:CekMAC} beserta \emph{flowchart} sesuai Gambar \ref{fig:Flowchart 1 MAC Address}.

\begin{figure} [ht] \centering
  % Nama dari file gambar yang diinputkan
  \includegraphics[scale=0.7]{gambar/program/1. MAC Address.png}
  % Keterangan gambar yang diinputkan
  \caption{\emph{Flowchart} Memeriksa MAC Address Pada ESP32}
  % Label referensi dari gambar yang diinputkan
  \label{fig:Flowchart 1 MAC Address}
\end{figure}

\newpage

Program ini menggunakan \emph{library} BluetoothSerial dari Henry Abrahamsen. \emph{Library} ini menyediakan fungsionalitas untuk mengontrol modul Bluetooth pada ESP32 \parencite{Abrahamsen_2023}. Selanjutnya akan dibuat objek SerialBT dari kelas BluetoothSerial. Objek ini digunakan untuk berkomunikasi melalui Bluetooth. 

Pada void setup() terdapat Serial.begin() yang digunakan untuk menginisialisasi komunikasi serial dengan kecepatan 115200 bps. Selanjutnya modul Bluetooth diinisialisasikan dengan nama perangkat ESP32\_Haris. Berikan delay selama 1 detik agar Bluetooth dapat diinisialisasikan dengan baik.

Array esp32Address dideklarasi pada awal fungsi void loop(). Array ini akan digunakan untuk menyimpan alamat dari Bluetooth ESP32. fungsi esp\_efuse\_mac\_get\_default() digunakan untuk mendapatkan alamat Bluetooth dan menyimpannya dalam array esp32Address \parencite{Systems_2023}. Alamat Bluetooth kemudian akan dicetak ke Serial Monitor menjadi bentuk yang dapat dibaca dengan menggunakan kode Serial.printf(). 

\subsection{Program Untuk Menerima Data String Melalui Bluetooth Pada ESP32}

Program ini dirancang untuk menerima data string melalui Bluetooth dan memisahkannya menjadi 2 bagian. Kemudian terdapat data yang dievaluasi untuk mengetahui apakah data tersebut sudah berubah menjadi integer. Berikut ini merupakan Program \ref{lst:ReceivedBluetooth} yang digunakan untuk menguji kemampuan ESP32 dalam menerima data string melalui Bluetooth beserta \emph{flowchart} sesuai Gambar \ref{fig:Flowchart 2 String}.

Program ini menggunakan \emph{library} BluetoothSerial dari Henry Abrahamsen. \emph{Library} ini menyediakan fungsionalitas untuk menerima data melalui koneksi Bluetooth dan mengolahnya \parencite{Abrahamsen_2023}. Selanjutnya akan dibuat objek SerialBT dari kelas BluetoothSerial. Objek ini selanjutnya digunakan untuk berkomunikasi melalui Bluetooth. Terdapat juga variabel maxspeed yang dideklarasikan sebagai suatu variabel yang menyimpan nilai kecepatan maksimum bertipe data integer.

Pada void setup() terdapat Serial.begin() yang digunakan untuk menginisialisasi komunikasi serial dengan kecepatan 115200 bps. Selanjutnya modul Bluetooth diinisialisasikan dengan nama perangkat ESP32\_Haris.

Pada fungsi void loop() akan berjalan terus-menerus setelah fungsi void setup(). Pertama-tama akan diperiksa apakah terdapat data yang tersedia untuk dibaca dari koneksi Bluetooth dengan menggunakan SerialBT.available(). Apabila terdapat data yang diterima maka data tersebut akan dimasukkan kedalam variabel receivedData yang bertipe string dengan menggunakan fungsi SerialBT.readStringUntil(). Data tersebut akan dibaca hingga menemukan karakter \emph{newline} ('\textbackslash n'). Data kemudian akan dipisahkan menjadi arah dan kecepatan. arah akan memisahkan data sebelum koma (',') dari receivedData. kecepatan akan memisahkan data setelah (',') hingga akhir receivedData. Tipe data dari variabel kecepatan ini kemudian akan diubah menjadi integer dan nilai tersebut dimasukkan ke dalam variabel maxspeed. Kemudian nilai pada variabel arah dan kecepatan akan ditampilkan ke \emph{Serial Monitor}. fungsi \emph{if} ditambahkan untuk menguji apakah variabel maxspeed sudah bertipe integer. Apabila nilai maxspeed kurang dari 20 maka akan ditampilkan \emph{boolean} \emph{True}, apabila lebih dari atau sama dengan 20 maka akan menampilkan \emph{boolean} \emph{False}.

\begin{figure} [ht] \centering
  % Nama dari file gambar yang diinputkan
  \includegraphics[scale=0.7]{gambar/program/2. Data String Bluetooth.png}
  % Keterangan gambar yang diinputkan
  \caption{\emph{Flowchart} Menerima Data String Melalui Bluetooth Pada ESP32}
  % Label referensi dari gambar yang diinputkan
  \label{fig:Flowchart 2 String}
\end{figure}

\subsection{Program Untuk Menerima Data JSON Melalui Bluetooth Pada ESP32}

Program ini dirancang untuk menerima data JSON melalui Bluetooth dan memisahkannya menjadi 2 bagian. Berikut merupakan program untuk menguji kemampuan ESP32 dalam menerima data JSON melalui Bluetooth yang dapat dilihat pada Program \ref{lst:ReceivedBluetoothJSON} beserta \emph{flowchart} sesuai Gambar \ref{fig:Flowchart 3 JSON}. 

Program ini menggunakan \emph{library} BluetoothSerial dari Henry Abrahamsen dan ArduinoJson dari Beno\^{\i}t Blanchon. \emph{Library} BluetoothSerial menyediakan fungsionalitas untuk menerima data melalui koneksi Bluetooth \parencite{Abrahamsen_2023}. Sedangkan \emph{library} ArduinoJson digunakan untuk menguraikan \emph{deserialize} data JSON yang diterima \parencite{blanchon2021mastering}. Selanjutnya akan dibuat objek SerialBT dari kelas BluetoothSerial. Objek ini akan digunakan untuk berkomunikasi melalui Bluetooth.

Pada void setup() terdapat Serial.begin() untuk menginisialisasi komunikasi serial dengan kecepatan 115200 bps. Selanjutnya modul Bluetooth diinisialisasi dengan nama perangkat ESP\_Haris.

Pada fungsi void loop() akan berjalan terus-menerus setelah fungsi void setup(). Pertama-tama akan diperiksa apakah terdapat data yang tersedia untuk dibaca dari koneksi Bluetooth dengan menggunakan SerialBT.available(). Kemudian data akan dibaca dari koneksi Bluetooth hingga terdapat karakter \emph{newline} ('\textbackslash n') dan menyimpannya dalam string json\_data. Objek DynamicJsonDocument selanjutnya akan dibuat dengan kapasitas 1024 byte yang berguna untuk menampung dokumen JSON yang akan diuraikan. Data JSON yang telah ditampung kemudian hasilnya disimpan dalam objek doc. Jika terdapat kesalahan dalam proses deserialisasi maka pesan kesalahan akan ditampilkan. Lalu terdapat fungsi \emph{if} yang digunakan untuk memberikan pesan \emph{"Failed to parse JSON"} apabila terdapat kesalahan pada proses deserialisasi. Apabila tidak terdapat kesalahan maka setiap nilai dari atribut JSON akan diekstrak dengan nama arah dan kecepatan serta menyimpannya dalam variabel bertipe string. Nilai yang diterima dari atribut JSON akan dicetak ke \emph{Serial Monitor}.

\begin{figure} [ht] \centering
  % Nama dari file gambar yang diinputkan
  \includegraphics[scale=0.7]{gambar/program/3. JSON Bluetooth.png}
  % Keterangan gambar yang diinputkan
  \caption{\emph{Flowchart} Menerima Data JSON Melalui Bluetooth Pada ESP32}
  % Label referensi dari gambar yang diinputkan
  \label{fig:Flowchart 3 JSON}
\end{figure}


\subsection{Program Untuk Menerima Data String Melalui \emph{Access Point} WiFi Pada ESP32}

Program ini dirancang sebagai server WiFi yang dapat menerima data dari \emph{client} yang terhubung dan mengolahnya. Data yang diterima akan diproses dan dipisahkan menjadi 2 variabel, yaitu variabel arah dan kecepatan. Kemudian nilai dari variabel kecepatan akan dimasukkan ke variabel maxspeed dan tipe datanya dievaluasi. Berikut merupakan Program \ref{lst:ReceivedWiFi} yang digunakan untuk menerima data string melalui \emph{Access Point} WiFi pada ESP32 beserta \emph{flowchart} sesuai Gambar \ref{fig:Flowchart 4 String}.

\begin{figure} [ht] \centering
  % Nama dari file gambar yang diinputkan
  \includegraphics[scale=0.7]{gambar/program/4. String WiFi.png}
  % Keterangan gambar yang diinputkan
  \caption{\emph{Flowchart} Menerima Data String Melalui \emph{Access Point} WiFi Pada ESP32}
  % Label referensi dari gambar yang diinputkan
  \label{fig:Flowchart 4 String}
\end{figure}

Program ini menggunakan \emph{library} WiFi yang menyediakan fungsionalitas untuk mengonfigurasi dan mengelola koneksi WiFi pada ESP32. Selain itu \emph{library} Arduino juga digunakan pada program ini yang menyediakan fungsi-fungsi dasar untuk pemrograman mikrokontroler. Variabel maxspeed dideklarasikan sebagai variabel yang menyimpan nilai integer yang kemudian digunakan untuk menyimpan nilai kecepatan maksimum. Variabel ssid akan digunakan untuk menyimpan nama jaringan WiFi yang akan dibuat dan variabel \emph{password} akan menyimpan nilai \emph{password} dari WiFi Server yang dibuat. Kedua variabel ini sangat penting agar ESP32 dapat dikonfigurasikan sebagai \emph{Access Point}. Selanjutnya objek server dibuat dari kelas WiFiServer untuk menangani koneksi pada port 80.

Pada void setup() terdapat Serial.begin() untuk menginisialisasi komunikasi serial dengan kecepatan 115200 bps. Lalu mengonfigurasi ESP32 sebagai \emph{Access Point} sesuai dengan SSID dan \emph{password} yang telah ditentukan. Apabila \emph{Access Point} telah dikonfigurasi maka IP \emph{Address} dari ESP32 akan disimpan pada variabel IP dan dicetak ke \emph{Serial Monitor}. server.begin() digunakan untuk memulai server pada port 80.

Fungsi void loop() akan berjalan terus-menerus setelah fungsi void setup() dijalankan. Pada fungsi ini terdapat WiFiClient yang digunakan untuk menguji apakah terdapat \emph{client} yang terhubung ke server. Jika terdapat \emph{client} yang terhubung maka akan masuk ke fungsi \emph{if}. Data yang dikirimkan akan dibaca secara terus-menerus melalui fungsi \emph{while}. Data yang diterima akan dimasukkan ke variabel receivedData hingga \emph{client} mengirimkan karakter \emph{newline} ('\textbackslash n'). Data kemudian akan dipisahkan menjadi arah dan kecepatan. arah akan memisahkan data sebelum koma (',') dari receivedData. kecepatan akan memisahkan data setelah (',') hingga akhir receivedData. Tipe data dari variabel kecepatan ini kemudian akan diubah menjadi integer dan nilai tersebut dimasukkan ke dalam variabel maxspeed. Kemudian nilai pada variabel arah dan kecepatan akan ditampilkan ke \emph{Serial Monitor}. fungsi \emph{if} ditambahkan untuk menguji apakah variabel maxspeed sudah bertipe integer. Apabila nilai maxspeed kurang dari 20 maka akan ditampilkan \emph{boolean} \emph{True}, apabila lebih dari atau sama dengan 20 maka akan menampilkan \emph{boolean} \emph{False}. Lalu koneksi akan ditutup setelah selesai membaca data dari \emph{client}.

\subsection{Program Untuk Menerima Data JSON Melalui \emph{Access Point} WiFi Pada ESP32}

Program ini dirancang untuk menerima data JSON melalui WiFi. ESP32 berperan sebagai \emph{Access Point} WiFi yang dapat menerima data JSON dari \emph{client} dan mencetaknya ke dalam \emph{Serial Monitor}. Berikut merupakan Program \ref{lst:ReceivedWiFiJSON} yang digunakan untuk menerima data JSON melalui \emph{Access Point} WiFi pada ESP32 beserta \emph{flowchart} sesuai Gambar \ref{fig:Flowchart 5 JSON}.

\begin{figure} [ht] \centering
  % Nama dari file gambar yang diinputkan
  \includegraphics[scale=0.7]{gambar/program/5. JSON WiFi.png}
  % Keterangan gambar yang diinputkan
  \caption{\emph{Flowchart} Menerima Data JSON Melalui \emph{Access Point} WiFi Pada ESP32}
  % Label referensi dari gambar yang diinputkan
  \label{fig:Flowchart 5 JSON}
\end{figure}

Program ini menggunakan \emph{library} WiFi yang menyediakan fungsionalitas untuk mengkonfigurasi dan mengelola koneksi WiFi pada ESP32. Selain itu \emph{library} Arduino juga digunakan pada program ini yang menyediakan fungsi-fungsi dasar untuk pemrograman mikrokontroler. \emph{Library} ArduinoJson dari Beno\^{\i}t Blanchon digunakan untuk mengolah data JSON. Variabel maxspeed yang dideklarasikan sebagai suatu variabel yang menyimpan nilai kecepatan maksimum yang diterima dari data JSON. Variabel ssid dan password digunakan untuk menyimpan nilai nama dan \emph{password} untuk \emph{access point} yang akan dibuat oleh ESP32. Lalu objek server dari kelas WiFiServer dibuat untuk menangani koneksi pada port 80.

Pada void setup() terdapat Serial.begin() yang digunakan untuk menginisialisasi komunikasi serial dengan kecepatan 115200 bps. Selanjutnya \emph{access point} akan dibuat sesuai dengan ssid dan password yang telah ditentukan sebelumnya. IP \emph{Address} kemudian akan ditampilkan pada \emph{Serial Monitor}. Akhirnya server akan dimulai pada port 80.

Fungsi void loop() akan berjalan terus-menerus setelah fungsi void setup() dijalankan. ESP32 akan mencoba untuk menerima koneksi dari \emph{client} melalui fungsi server.available(). Jika terdapat \emph{client} yang terhubung maka ESP32 akan memeriksa apakah \emph{client} masih tetap terhubung. Jika terdapat data yang tersedia maka data akan dibaca hingga terdapat karakter \emph{newline}('\textbackslash n'). Selanjutnya objek doc akan dibuat dari kelas DynamicJsonDocument dengan kapasitas 1024 \emph{byte}. Data yang didapatkan kemudian akan dipecah dengan menggunakan fungsi deserializeJson.

Jika terjadi kesalahan saat mengurai JSON maka program akan mencetak pesan \emph{error}. Jika \emph{parsing} JSON berhasil dilakukan, maka nilai dari JSON akan diakses. Data JSON akan diurai dan disimpan kedalam variabel arah serta kecepatan. Selanjutnya program akan menampilkan variabel tersebut ke \emph{Serial Monitor}. Setelah data JSON selesai diproses maka koneksi dengan \emph{client} akan ditutup.

\subsection{Program Untuk Menguji Rangkaian Kontrol ESP32}

Program ini dirancang untuk menguji rangkaian kontrol ESP32. Kedua motor akan diuji dan dikendalikan melalui modul ESP32 dan driver motor H-Bridge. Berikut program \ref{lst:MotorTest} yang digunakan untuk menguji rangkaian kontrol ESP32 beserta \emph{flowchart} sesuai Gambar \ref{fig:Flowchart 6 Uji Kontrol}.

\emph{Library} Arduino digunakan pada program ini. \emph{Library} ini menyediakan berbagai fungsi-fungsi dasar untuk pemrograman mikrokontroler. pwmPin1, dir1, dan dir2 merupakan variabel untuk mengatur kerja dari motor kiri. pwmPin1 didefinisikan dengan nilai 5 yang kemudian akan terhubung dengan GPIO5 pada ESP32, dir1 didefinisikan dengan nilai 18 yang kemudian akan terhubung dengan GPIO18 pada ESP32, dan dir2 yang didefinisikan dengan nilai 19 yang kemudian akan terhubung dengan GPIO19 pada ESP32. Selanjutnya terdapat pwmPin2, dir3, dan dir4 yang merupakan variabel yang digunakan untuk mengatur kerja dari motor kanan. pwmPin2 didefinisikan dengan nilai 25 yang kemudian akan terhubung dengan GPIO25 pada ESP32, dir3 didefinisikan dengan nilai 32 yang kemudian akan terhubung dengan GPIO32 pada ESP32, dan dir4 yang didefinisikan dengan nilai 33 yang kemudian akan terhubung dengan GPIO33 pada ESP32. Lalu variabel pwmChannel1, pwmChannel2, freq, dan res dideklarasikan. pwmChannel1 dan pwmChannel2 merupakan nomor saluran PWM yang digunakan. Variabel freq merupakan frekuensi PWM yang diatur menjadi 15kHz untuk meminimalisir \emph{noise} pada motor ketika bekerja. Variabel res digunakan untuk mengatur resolusi dari PWM yang diatur menjadi 8-bit. Variabel PWM1\_DutyCycle akan menyimpan siklus tugas yang kemudian akan digunakan untuk mengendalikan kecepatan motor dan variabel maxspeed merupakan nilai dari kecepatan motor maksimal.

\begin{figure} [ht] \centering
  % Nama dari file gambar yang diinputkan
  \includegraphics[scale=0.7]{gambar/program/6. Uji Kontrol.png}
  % Keterangan gambar yang diinputkan
  \caption{\emph{Flowchart} Menguji Rangkaian Kontrol ESP32}
  % Label referensi dari gambar yang diinputkan
  \label{fig:Flowchart 6 Uji Kontrol}
\end{figure}

Terdapat fungsi void setup() yang dieksekusi satu kali pada saat program pertama kali dijalankan. Pada fungsi ini terdapat dir1, dir2, dir3, dan dir4 yang diatur sebagai pin \emph{output}. ledcSetup merupakan fungsi dari \emph{library} ledc yang digunakan untuk mengonfigurasi saluran PWM. Fungsi ini memerlukan 3 variabel, yaitu pwmChannel, freq, dan res. Selanjutnya terdapat fungsi ledcAttachPin yang merupakan fungsi dari \emph{library} ledc yang digunakan untuk menghubungkan suatu pin dengan saluran PWM tertentu. Fungsi ini memerlukan 2 variabel, yaitu pin PWM dan juga PWM Channel. Secara keseluruhan, blok kode ini digunakan untuk melakukan konfigurasi pin-pin pada mikrokontroler ESP32 sehingga motor dapat dikendalikan. Pin-pin yang diatur sebagai \emph{output} akan digunakan untuk mengatur arah putar motor. Sedangkan saluran PWM akan digunakan untuk mengatur kecepatan motor.

Pada fungsi void loop() terdapat serangkaian perintah yang mengatur gerakan kursi roda dengan menggunakan motor DC. Gerakan ini akan diulang secara terus-menerus. Hal ini berguna untuk menguji rangkaian kontrol pada ESP32. Setiap gerakan diawali dengan \emph{softstart} dan diakhiri dengan \emph{softstop}. Pada gerakan maju, kedua motor akan bergerak maju. Selama nilai PWM1\_DutyCycle kurang dari maxspeed, maka nilai PWM1\_DutyCycle akan ditambahkan secara bertahap hingga mencapai nilai maxspeed yang telah ditentukan. Selama itu juga kecepatan putar motor akan bertambah secara bertahap. Penambahan nilai PWM1\_DutyCycle akan ditunda selama 10 milidetik. Setelah PWM1\_DutyCycle mencapai nilai maxspeed maka kode selanjutnya akan ditunda selama 5 detik yang mengakibatkan motor akan berputar maju dengan kecepatan maksimum selama 5 detik. Setelah 5 detik, apabila nilai dari PWM1\_DutyCycle lebih dari 0 maka akan dilakukan \emph{softstop}. Arah putar kedua motor tetap diatur maju. Nilai PWM1\_DutyCycle akan diturunkan secara bertahap. Pengurangan nilai PWM1\_DutyCycle akan ditunda selama 10 milidetik. Apabila nilai PWM1\_DutyCycle sudah mencapai nilai 0 maka kode selanjutnya akan ditunda selama 1 detik. Langkah yang serupa akan dilakukan untuk setiap gerakan. Setiap gerakan memiliki fase \emph{softstart} dan \emph{softstop} yang bertujuan untuk memberikan perubahan kecepatan yang halus dan menghindari perubahan secara tiba-tiba. Penggunaan variabel PWM1\_DutyCycle digunakan untuk mengatur siklus kerja PWM untuk sebagai kontrol kecepatan yang fleksibel. fungsi \emph{delay} digunakan untuk memberikan jeda waktu antara setiap gerakan untuk mengatur durasi gerakan dan memberikan waktu bagi motor kursi roda untuk menyelesaikan setiap gerakan sebelum beralih ke gerakan berikutnya. Pada fungsi mundur, kedua roda akan berputar mundur. Pada fungsi belok kanan, roda kiri akan berputar maju sedangkan roda kanan tidak berputar sehingga kursi roda dapat berbelok ke arah kanan. Pada fungsi belok kiri, motor kiri tidak berputar namun roda kanan akan berputar maju sehingga kursi roda dapat berbelok ke arah kiri. Pada fungsi stop, kedua roda tidak akan berputar. Proses ini terus berulang didalam \emph{loop}, sehingga kursi roda akan melakukan gerakan berulang sesuai dengan logika yang telah diatur.


\subsection{Program Kontrol Motor Kursi Roda Melalui Bluetooth}

Program ini dirancang untuk mengendalikan motor DC dengan memproses perintah yang diterima melalui koneksi Bluetooth. Setelah menerima data arah dan kecepatan, maka program ini akan mengatur motor sesuai dengan perintah yang diterima. Berikut merupakan Program \ref{lst:MotorBluetooth} yang digunakan untuk mengontrol motor kursi roda melalui Bluetooth beserta \emph{flowchart} sesuai Gambar \ref{fig:Flowchart 7 Kontrol Bluetooth}.

Program ini menggunakan \emph{library} BluetoothSerial dari Henry Abrahamsen dan juga \emph{library} Arduino. \emph{Library} BluetoothSerial menyediakan fungsionalitas untuk menerima data melalui koneksi Bluetooth \parencite{Abrahamsen_2023}. \emph{Library} Arduino menyediakan berbagai fungsi dasar untuk pemrograman mikrokontroler. pwmPin1, dir1, dan dir2 merupakan variabel yang digunakan untuk mengatur kerja dari motor kiri. pwmPin1 didefinisikan dengan nilai 5 yang kemudian akan terhubung dengan GPIO5 pada ESP32, dir1 didefinisikan dengan nilai 18 yang kemudian akan terhubung dengan GPIO18 pada ESP32, dan dir2 didefinisikan dengan nilai 19 yang kemudian akan terhubung dengan GPIO19 pada ESP32. Selanjutnya terdapat pwmPin2, dir3, dan dir4 merupakan variabel yang akan digunakan untuk mengatur kerja dari motor kanan. pwmPin2 didefinisikan dengan nilai 25 yang kemudian akan terhubung dengan GPIO25 pada ESP32, dir3 didefinisikan dengan nilai 32 yang kemudian akan terhubung dengan GPIO32, dan dir4 yang didefinisikan dengan nilai 33 yang kemudian akan terhubung dengan GPIO33 pada ESP32. Array stdir digunakan untuk menyimpan keadaan arah motor pada setiap langkah. Lalu variabel pwmChannel1, pwmChannel2. req, dan res dideklarasikan. pwmChannel1 dan pwmChannel2 merupakan nomor saluran PWM yang digunakan. Variabel freq merupakan frekuensi PWM yang diatur menjadi 15 kHz untuk meminimalisir \emph{noise} pada motor ketika bekerja. Variabel res digunakan untuk mengatur resolusi dari PWM yang diatur menjadi 8-bit. Variabel PWM1\_DutyCycle akan menyimpan siklus tugas yang kemudian akan digunakan untuk mengendalikan kecepatan motor. Variabel maxspeed digunakan untuk mengatur kecepatan maksimal dari motor dan variabel turnspeed digunakan untuk kecepatan maksimal ketika kursi roda berbelok.

\begin{figure} [ht] \centering
  % Nama dari file gambar yang diinputkan
  \includegraphics[scale=0.7]{gambar/program/7. Kontrol Motor Bluetooth.png}
  % Keterangan gambar yang diinputkan
  \caption{\emph{Flowchart} Kontrol Motor Kursi Roda Melalui Bluetooth}
  % Label referensi dari gambar yang diinputkan
  \label{fig:Flowchart 7 Kontrol Bluetooth}
\end{figure}

Fungsi void setup() dieksekusi satu kali pada saat program pertama kali dijalankan. Pada fungsi ini objek SerialBT diinisialisasi sebagai Bluetooth Serial dengan nama ESP32\_Haris. dir1, dir2, dir3, dan dir4 diatur sebagai \emph{output}. ledcSetup merupakan fungsi dari \emph{library} ledc yang digunakan untuk menghubungkan suatu pin dengan saluran PWM tertentu. Fungsi ini memerlukan 2 variabel, yaitu pin PWM dan juga PWM \emph{Channel}. Secara keseluruhan blok kode ini digunakan untuk melakukan konfigurasi pin-pin pada mikrokontroler ESP32 sehingga motor dapat dikendalikan. Pin-pin yang diatur sebagai \emph{output} akan digunakan untuk mengatur arah putar motor. Sedangkan saluran PWM akan digunakan untuk mengatur kecepatan putar motor.

Pada fungsi void loop() terdapat perintah untuk memeriksa ketersediaan data dari Bluetooth dengan SerialBT.acailable(). Jika terdapat \emph{client} yang terhubung maka data akan dibaca hingga \emph{newline} ('\textbackslash n'). Data yang diterima kemudian akan diurai dan dimasukkan kedalam variabel arah dan kecepatan. Tipe data dari variabel kecepatan akan diubah dari string menjadi integer dengan menggunakan fungsi toInt(). Nilai dari maxspeed dan turnspeed kemudian akan diatur sesuai dengan perintah. Kontrol arah gerak motor menggunakan pernyataan kondisional (\emph{if-else}) berdasarkan variabel arah yang diterima. Pada setiap kondisi terdapat perulangan \emph{while}. Hal ini dilakukan agar perubahan kecepatan putar motor tidak terjadi secara tiba-tiba. digitalWrite digunakan untuk menetapkan arah putar motor. ledcWrite digunakan untuk mengendalikan kecepatan motor. Perulangan ini terus dilakukan selama ESP32 masih menerima data melalui Bluetooth.

\subsection{Program Kontrol Motor Kursi Roda Melalui \emph{Access Point} WiFi}

Program ini dirancang untuk mengendalikan motor DC dengan memproses perintah yang diterima melalui koneksi WiFi. ESP32 bertindak sebagai \emph{Access Point}. Setelah menerima data arah maka program ini akan mengatur motor sesuai dengan perintah yang diterima. Berikut merupakan Program \ref{lst:MotorWiFiAP} yang digunakan untuk mengontrol motor kursi roda melalui WiFi beserta \emph{flowchart} sesuai Gambar \ref{fig:Flowchart 8 Kontrol WiFi}.

\begin{figure} [ht] \centering
  % Nama dari file gambar yang diinputkan
  \includegraphics[scale=0.7]{gambar/program/8. Kontrol Motor WiFi.png}
  % Keterangan gambar yang diinputkan
  \caption{\emph{Flowchart} Kontrol Motor Kursi Roda Melalui \emph{Access Point} WiFi}
  % Label referensi dari gambar yang diinputkan
  \label{fig:Flowchart 8 Kontrol WiFi}
\end{figure}

Program ini menggunakan \emph{library} WiFi yang menyediakan fungsionalitas untuk mengkonfigurasi dan mengelola koneksi WiFi pada ESP32. Selain itu \emph{library} Arduino juga digunakan pada program ini yang menyediakan fungsi-fungsi dasar untuk pemrograman mikrokontroler. Variabel ssid dan password digunakan untuk menyimpan nilai nama dan \emph{password} untuk \emph{access point} yang akan dibuat oleh ESP32. Lalu objek server dari kelas WiFiServer dibuat untuk menangani koneksi pada port 80.

pwmPin1, dir1, dan dir2 merupakan variabel yang digunakan untuk mengatur kerja dari motor kiri. pwmPin1 didefinisikan dengan nilai 5 yang kemudian akan terhubung dengan GPIO5 pada ESP32, dir1 didefinisikan dengan nilai 18 yang kemudian akan terhubung dengan GPIO18 pada ESP32, dan dir2 didefinisikan dengan nilai 19 yang kemudian akan terhubung dengan GPIO19 pada ESP32. Selanjutnya terdapat pwmPin2, dir3, dan dir4 merupakan variabel yang akan digunakan untuk mengatur kerja dari motor kanan. pwmPin2 didefinisikan dengan nilai 25 yang kemudian akan terhubung dengan GPIO25 pada ESP32, dir3 didefinisikan dengan nilai 32 yang kemudian akan terhubung dengan GPIO32, dan dir4 yang didefinisikan dengan nilai 33 yang kemudian akan terhubung dengan GPIO33 pada ESP32. Array stdir digunakan untuk menyimpan keadaan arah motor pada setiap langkah. Lalu variabel pwmChannel1, pwmChannel2. req, dan res dideklarasikan. pwmChannel1 dan pwmChannel2 merupakan nomor saluran PWM yang digunakan. Variabel freq merupakan frekuensi PWM yang diatur menjadi 15 kHz untuk meminimalisir \emph{noise} pada motor ketika bekerja. Variabel res digunakan untuk mengatur resolusi dari PWM yang diatur menjadi 8-bit. Variabel PWM1\_DutyCycle akan menyimpan siklus tugas yang kemudian akan digunakan untuk mengendalikan kecepatan motor. Variabel maxspeed digunakan untuk mengatur kecepatan maksimal dari motor dan variabel turnspeed digunakan untuk kecepatan maksimal ketika kursi roda berbelok.

Fungsi void setup() dieksekusi satu kali pada saat program pertama kali dijalankan. Pada fungsi ini terdapat Serial.begin() yang digunakan untuk menginisialisasi komunikasi serial dengan kecepatan 115200 bps. Selanjutnya \emph{access point} akan dibuat sesuai dengan ssid dan \emph{password} yang telah ditentukan. IP \emph{Address} kemudian akan ditampilkan pada \emph{Serial Monitor}. Lalu server akan dimulai pada port 80. dir1, dir2, dir3, dan dir4 diatur sebagai \emph{output}. ledcSetup merupakan fungsi dari \emph{library} ledc yang digunakan untuk menghubungkan suatu pin dengan saluran PWM tertentu. Fungsi ini memerlukan 2 variabel, yaitu pin PWM dan juga PWM \emph{Channel}. Secara keseluruhan blok kode ini digunakan untuk melakukan konfigurasi pin-pin pada mikrokontroler ESP32 sehingga motor dapat dikendalikan. Pin-pin yang diatur sebagai \emph{output} akan digunakan untuk mengatur arah putar motor. Sedangkan saluran PWM akan digunakan untuk mengatur kecepatan putar motor.

Pada fungsi void loop() terdapat perintah untuk memeriksa ketersediaan koneksi \emph{client}. Jika terdapat \emph{client} yang terhubung maka ESP32 akan memeriksa apakah \emph{client} masih tetap terhubung. Jika terdapat data yang tersedia maka data tersebut akan dibaca hingga terdapat karakter \emph{newline} ('\textbackslash n'). Data yang diterima kemudian akan diurai dan dimasukkan kedalam variabel arah. Kontrol arah gerak motor menggunakan pernyataan kondisional \emph{if-else} berdasarkan variabel arah yang diterima. Pada setiap kondisi terdapat perulangan \emph{while}. Hal ini dilakukan agar perubahan kecepatan motor tidak terjadi secara tiba-tiba. digitalWrite digunakan untuk menetapkan arah putar motor. ledcWrite digunakan untuk mengendalikan kecepatan motor. Perulangan ini terus dilakukan selama ESP32 masih menerima data melalui WiFi.

\subsection{Program Untuk Mengirim Data String Melalui Bluetooth}

Program ini menggunakan bahasa pemrograman Python. Program ini dirancang untuk mengirim data string ke perangkat ESP32 melalui Bluetooth. Berikut merupakan Program \ref{lst:SendBluetooth} yang digunakan untuk mengirim data string melalui Bluetooth beserta \emph{flowchart} sesuai Gambar \ref{fig:Flowchart 9 Mengirim String Bluetooth}.

\begin{figure} [ht] \centering
  % Nama dari file gambar yang diinputkan
  \includegraphics[scale=0.65]{gambar/program/9. Mengirim Data String Bluetooth.png}
  % Keterangan gambar yang diinputkan
  \caption{\emph{Flowchart} Mengirim Data String Melalui Bluetooth}
  % Label referensi dari gambar yang diinputkan
  \label{fig:Flowchart 9 Mengirim String Bluetooth}
\end{figure}

Program ini menggunakan \emph{library} Bluetooth untuk berinteraksi dengan perangkat lain melalui Bluetooth dan juga \emph{library} datetime yang digunakan untuk mendapatkan informasi waktu secara \emph{realtime}. Lalu menginisialisasi MAC \emph{Address} dari perangkat ESP32 yang akan dikendalikan. Objek soket Bluetooth dibuat menggunakan fungsi BluetoothSocket lalu menghubungkannya dengan perangkat ESP32 melalui MAC \emph{Address} dan nomor \emph{channel}. Pada umumnya, \emph{channel} 1 digunakan untuk profil \emph{Serial Port Profile} (SPP)

Kemudian program berjalan pada perulangan yang tak terbatas. Pengguna diminta untuk memasukkan nilai arah dan kecepatan melalui \emph{input}. Data tersebut kemudian akan digabungkan menjadi satu pesan dan dikirimkan ke perangkat ESP32 melalui soket Bluetooth. Apabila pengguna memasukkan "q" sebagai nilai arah maupun kecepatan, maka perulangan akan dihentikan dan soket Bluetooth akan ditutup.

\subsection{Program Untuk Mengirim Data String Melalui WiFi}

Program ini menggunakan bahasa pemrograman Python. Program ini dirancang untuk mengirim data string ke perangkat ESP32 melalui WiFi. ESP32 akan bertindak sebagai \emph{Access Point}. Berikut merupakan Program \ref{lst:SendWiFiAP} yang digunakan untuk mengirimkan data string melalui WiFi beserta \emph{flowchart} sesuai Gambar \ref{fig:Flowchart 10 Mengirim String WiFi}.

\begin{figure} [ht] \centering
  % Nama dari file gambar yang diinputkan
  \includegraphics[scale=0.7]{gambar/program/10. Mengirim Data String WiFi.png}
  % Keterangan gambar yang diinputkan
  \caption{\emph{Flowchart} Mengirim Data String Melalui WiFi}
  % Label referensi dari gambar yang diinputkan
  \label{fig:Flowchart 10 Mengirim String WiFi}
\end{figure}

Program ini menggunakan \emph{library} socket, time, dan datetime. \emph{Library} socket menyediakan antarmuka untuk membuat dan berinteraksi dengan soket sehingga dapat membuat koneksi dan mengirim ataupun menerima data melalui jaringan. \emph{Library} time menyediakan fungsi-fungsi terkait waktu dan pengukuran waktu. Fungsi yang sering digunakan adalah time.sleep() yang berguna untuk menghentikan eksekusi program selama jumlah waktu tertentu. \emph{Library} datetime digunakan untuk mendapatkan informasi waktu secara \emph{realtime}. Selanjutnya IP \emph{Address} dari \emph{Access Point} ESP32 dimasukkan kedalam variabel host. Nomor port yang digunakan untuk koneksi akan dimasukkan kedalam variabel port. Umumnya menggunakan port 80.

Objek soket kemudian dibuat menggunakan socket.socket() dengan alamat IPv4 dan tipe soket streaming. Ini menunjukkan bahwa program ini akan menggunakan protokol TCP/IP untuk koneksi. Kemudian mencoba untuk membuat koneksi ke server yang telah ditentukan sesuai IP \emph{Address} host dan nomor port.

Kemudian program akan berjalan pada perulangan yang tak terbatas. Pengguna diminta untuk memasukkan nilai arah dan kecepatan melalui \emph{input}. Data tersebut kemudian digabungkan menjadi 1 pesan. Pesan kemudian dikonversi menjadi \emph{byte} menggunakan \emph{encoding} UTF-8. Setelah dikonversi maka pesan tersebut akan dikirimkan ke server. Apabila pengguna memasukkan "q" sebagai nilai arah maupun kecepatan, maka perulangan akan dihentikan dan program akan menutup koneksi soket.

\subsection{Program Untuk Mengirim Data JSON Melalui Bluetooth}

Program ini menggunakan bahasa pemrograman Python. Program ini dirancang untuk mengirimkan data JSON melalui Bluetooth. Berikut merupakan Program \ref{lst:SendBluetoothJSON} yang digunakan untuk mengirim data JSON melalui Bluetooth beserta \emph{flowchart} sesuai Gambar \ref{fig:Flowchart 11 Mengirim JSON Bluetooth}.

\begin{figure} [ht] \centering
  % Nama dari file gambar yang diinputkan
  \includegraphics[scale=0.7]{gambar/program/11. Mengirim Data JSON Bluetooth.png}
  % Keterangan gambar yang diinputkan
  \caption{\emph{Flowchart} Mengirim Data JSON Melalui Bluetooth}
  % Label referensi dari gambar yang diinputkan
  \label{fig:Flowchart 11 Mengirim JSON Bluetooth}
\end{figure}

Program ini menggunakan \emph{library} Bluetooth untuk berinteraksi dengan perangkat lain melalui Bluetooth dan \emph{library} JSON untuk mengelola data dalam format JSON. Alamat Bluetooth dari perangkat ESP32 yang akan dikendalikan kemudian dimasukkan kedalam suatu variabel. Objek soket Bluetooth kemudian dibuat dengan menggunakan protokol \emph{Radio Frequen-cy Communication}(RFCOMM). RFCOMM merupakan protokol yang umum digunakan untuk komunikasi serial melalui Bluetooth. Soket Bluetooth kemudian dihubungkan ke perangkat ESP32 menggunakan alamat Bluetooth dan nomor \emph{channel}. Pada umumnya \emph{channel} 1 digunakan untuk profil SPP (\emph{Serial Port Profile}).

Kemudian program akan berjalan dalam perulangan yang tak terbatas. Pengguna diminta untuk memasukkan nilai arah dan kecepatan melalui \emph{input}. Objek JSON kemudian dibuat untuk menyimpan data yang telah dimasukkan. Objek JSON kemudian dikonversi menjadi string menggunakan json.dumps(). String JSON kemudian akan dikirimkan melalui koneksi Bluetooth menggunakan sock.send(). Apabila pengguna memasukkan "q" sebagai nilai arah maupun kecepatan, maka perulangan akan dihentikan dan program akan menutup koneksi soket.

\subsection{Program Untuk Mengirim Data JSON Melalui WiFi}

Program ini dibuat dengan menggunakan bahasa pemrograman Python. Program ini dirancang untuk mengirim data JSON ke perangkat ESP32 melalui WiFi. ESP32 akan bertindak sebagai \emph{Access Point}. Berikut merupakan Program \ref{lst:SendWiFiJSON} yang digunakan untuk mengirimkan data JSON melalui WiFi beserta \emph{flowchart} sesuai Gambar \ref{fig:Flowchart 12 Mengirim JSON WiFi}.

\begin{figure} [ht] \centering
  % Nama dari file gambar yang diinputkan
  \includegraphics[scale=0.7]{gambar/program/12. Mengirim Data JSON WiFi.png}
  % Keterangan gambar yang diinputkan
  \caption{\emph{Flowchart} Mengirim Data JSON Melalui WiFi}
  % Label referensi dari gambar yang diinputkan
  \label{fig:Flowchart 12 Mengirim JSON WiFi}
\end{figure}

Program ini menggunakan \emph{library} socket, time, json, dan datetime. \emph{Library} socket menyediakan antarmuka untuk membuat dan berinteraksi dengan soket sehingga dapat membuat koneksi dan mengirim ataupun menerima data melalui jaringan. \emph{Library} time menyediakan fungsi-fungsi terkait waktu dan pengukuran waktu. Fungsi yang sering digunakan adalah time.sleep() yang berguna untuk menghentikan eksekusi program selama jumlah waktu tertentu. \emph{Library} datetime digunakan untuk mendapatkan informasi waktu secara \emph{realtime}. \emph{Library} JSON digunakan untuk mengelola data dalam bentuk JSON. Selanjutnya IP \emph{Address} dari \emph{Access Point} ESP32 dimasukkan kedalam variabel host. Nomor port yang digunakan untuk koneksi akan dimasukkan kedalam variabel port. Umumnya menggunakan port 80.

Kemudian program akan berjalan dalam perulangan yang tak terbatas. Pengguna diminta untuk memasukkan nilai arah dan kecepatan melalui \emph{input}. Objek JSON kemudian dibuat untuk menyimpan data yang telah dimasukkan. Objek JSON kemudian dikonversi menjadi string menggunakan json.dumps(). String JSON kemudian akan dikonversi menjadi \emph{byte} menggunakan \emph{encoding} UTF-8. Setelah dikonversi maka akan dikirimkan melalui koneksi WiFi menggunakan s.send(). Apabila pengguna memasukkan "q" sebagai nilai arah maupun kecepatan, maka perulangan akan dihentikan dan program akan menutup koneksi soket.