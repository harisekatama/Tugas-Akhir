\begin{center}
  \Large
  \textbf{KATA PENGANTAR}
\end{center}

\addcontentsline{toc}{chapter}{KATA PENGANTAR}

\vspace{2ex}

% Ubah paragraf-paragraf berikut dengan isi dari kata pengantar

Puji dan syukur kehadirat Tuhan Yang Maha Esa, atas segala rahmat dan karunia-Nya,
sehingga penulis dapat menyelesaikan penelitian ini yang berjudul
Prediksi Jumlah Kalori Yang Terbakar Saat Olahraga \emph{Pull-Up} Berbasis CNN Dengan Menggunakan Jetson Nano.


Penelitian ini disusun dalam rangka pemenuhan Tugas Akhir sebagai syarat
kelulusan Mahasiswa ITS. Oleh karena itu, penulis mengucapkan banyak terima kasih kepada

\begin{enumerate}[nolistsep]

  \item Bapak Dr.Supeno Mardi Susiko Nugroho, ST.,MT, selaku Kepala Departemen Teknik Komputer, Fakultas Elektro dan Informatika Cerdas, Institut Teknologi Sepuluh Nopember

  \item Bapak Dr. Eko Mulyanto Yuniarno, S.T., M.T. selaku Dosen Pembimbing telah memberikan arahan selama pengerjaan tugas akhir ini

  \item Bapak Arief Kurniawan, S.T., M.T selaku dosen penguji I dan Ibu Dr. Susi Juniastuti, S.T., M.Eng selaku dosen penguji II yang telah memberikan saran dan revisi agar pengerjaan Buku Tugas Akhir ini dapat menjadi lebih baik

  \item Bapak-Ibu dosen pengajar Departemen Teknik Komputer, atas ilmu dan pengajaran yang telah diberikan kepada penulis selama ini

  \item Farel Jevon, S.T., Paschalis Seto Wicaksono, S.T., Felix Titus Setiawan, S.T., dan I Gusti Komang Agung Wiguna, S.T. yang telah memberikan inspirasi sehingga penelitian ini dilaksanakan

  \item I Putu Krisna Erlangga, Muh. Khaeral Azzam, Moh. Iqbal Fatchurozi, Batrisyia Zahrani Ananto, Evandrew Reynald Collin, Dimas Triananda Murti Putra, Ruky Augusta Gautama, Muh. Rezky Firdaus Irwan, Raka Zein Akbar, Priansa Putra Jaya Wardana dan teman - teman lab B300 lainnya yang telah menemani setiap kegiatan penelitian
  
  \item Teman - teman Departemen Teknik Komputer

\end{enumerate}

Akhir kata, semoga penelitian ini dapat memberikan manfaat kepada banyak pihak,
penulis menyadari jika skripsi ini masih belum sempurna, dikarenakan keterbatasan ilmu yang dimiliki. 
Untuk itu penulis mengharapkan saran dan kritik yang bersifat membangun kepada penulis untuk menuai hasil yang lebih baik lagi.

\begin{flushright}
  \begin{tabular}[b]{c}
    \place{}, \MONTH{} \the\year{} \\
    \\
    \\
    \\
    \\
    \name{}
  \end{tabular}
\end{flushright}
