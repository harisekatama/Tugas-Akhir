\begin{center}
  \Large
  \textbf{KATA PENGANTAR}
\end{center}

\addcontentsline{toc}{chapter}{KATA PENGANTAR}

\vspace{2ex}

% Ubah paragraf-paragraf berikut dengan isi dari kata pengantar

Puji dan syukur kehadirat Tuhan Yang Maha Esa, atas segala rahmat dan karunia-Nya,
sehingga penulis dapat menyelesaikan penelitian ini yang berjudul
"\tatitle".


Penelitian ini disusun dalam rangka pemenuhan Tugas Akhir sebagai syarat
kelulusan Mahasiswa ITS. Oleh karena itu, penulis mengucapkan banyak terima kasih kepada

\begin{enumerate}[nolistsep]

  \item Bapak Dr.Supeno Mardi Susiko Nugroho, ST.,MT, selaku Kepala Departemen Teknik Komputer, Fakultas Teknologi Elektro dan Informatika Cerdas, Institut Teknologi Sepuluh Nopember

  \item Bapak Dr. Eko Mulyanto Yuniarno, S.T., M.T. selaku Dosen Pembimbing telah memberikan arahan selama pengerjaan tugas akhir ini

  \item Bapak Dion Hayu Fandiantoro, S.T., M.Eng. selaku dosen penguji I, Bapak Dr. Arief Kurniawan, S.T., M.T. selaku dosen penguji II dan Bapak Arta Kusuma Hernanda, S.T., M.T. selaku dosen penguji III yang telah memberikan saran dan revisi agar pengerjaan Buku Tugas Akhir ini dapat menjadi lebih baik

  \item Bapak-Ibu dosen pengajar Departemen Teknik Komputer, atas ilmu dan pengajaran yang telah diberikan kepada penulis selama ini 
  
  \item Nenek, bibi dan ibu yang senantiasa membantu serta membiayai pendidikan sedari kecil hingga sarjana
  
  \item Teman - teman lab B300 dan B201 serta teman - teman Departemen Teknik Komputer lainnya

\end{enumerate}

Akhir kata, semoga penelitian ini dapat memberikan manfaat kepada banyak pihak,
penulis menyadari jika skripsi ini masih belum sempurna, dikarenakan keterbatasan ilmu yang dimiliki. 
Untuk itu penulis mengharapkan saran dan kritik yang bersifat membangun kepada penulis untuk menuai hasil yang lebih baik lagi.

\begin{flushright}
  \begin{tabular}[b]{c}
    \place{}, \MONTH{} \the\year{} \\
    \\
    \\
    \\
    \\
    \name{}
  \end{tabular}
\end{flushright}
