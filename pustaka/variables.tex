% Atur variabel berikut sesuai namanya

% nama
\newcommand{\name}{I Putu Haris Setiadi Ekatama}
\newcommand{\authorname}{Setiadi Ekatama, I Putu Haris}
\newcommand{\nickname}{Haris}
\newcommand{\advisor}{Dr. Eko Mulyanto Yuniarno, S.T., M.T.}
% \newcommand{\coadvisor}{-}
\newcommand{\examinerone}{Dion Hayu Fandiantoro, S.T., M.Eng.}
\newcommand{\examinertwo}{Dr. Arief Kurniawan, S.T., M.T.}
\newcommand{\examinerthree}{Arta Kusuma Hernanda, S.T., M.T.}
\newcommand{\headofdepartment}{Dr. Supeno Mardi Susiki Nugroho, S.T., M.T.}

% identitas
\newcommand{\nrp}{0721 19 4000 0046}
\newcommand{\advisornip}{19680601 199512 1 009}
% \newcommand{\coadvisornip}{-}
\newcommand{\examineronenip}{19942020 11064}
\newcommand{\examinertwonip}{19740907 200212 1 001}
\newcommand{\examinerthreenip}{19962023 11024}
\newcommand{\headofdepartmentnip}{19700313 199512 1 001}

% judul
\newcommand{\tatitle}{PERANCANGAN SISTEM KONTROL MOTOR KURSI RODA SECARA NIRKABEL BERBASIS ESP32}
\newcommand{\engtatitle}{\emph{DESIGNING A WIRELESS CONTROL SYSTEM FOR WHEELCHAIR MOTORS BASED ON ESP32}}

% tempat
\newcommand{\place}{Surabaya}

% jurusan
\newcommand{\studyprogram}{Teknik Komputer}
\newcommand{\engstudyprogram}{Computer Engineering}

% fakultas
\newcommand{\faculty}{Teknologi Elektro dan Informatika Cerdas}
\newcommand{\engfaculty}{Intelligent Electrical and Informatics Technology}

% singkatan fakultas
\newcommand{\facultyshort}{FTEIC}
\newcommand{\engfacultyshort}{F-ELECTICS}

% departemen
\newcommand{\department}{Teknik Komputer}
\newcommand{\engdepartment}{Computer Engineering}

% kode mata kuliah
\newcommand{\coursecode}{EC234801}